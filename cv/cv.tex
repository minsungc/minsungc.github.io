\documentclass[12pt]{article}

%----------------------------------------------------------------------------------------
%	FONT
%----------------------------------------------------------------------------------------

% % fontspec allows you to use TTF/OTF fonts directly
% \usepackage{fontspec}
% \defaultfontfeatures{Ligatures=TeX}

% % modified for ShareLaTeX use
% \setmainfont[
% SmallCapsFont = Fontin-SmallCaps.otf,
% BoldFont = Fontin-Bold.otf,
% ItalicFont = Fontin-Italic.otf
% ]
% {Fontin.otf}

%----------------------------------------------------------------------------------------
%	PACKAGES
%----------------------------------------------------------------------------------------
\usepackage{url, amsmath}
\usepackage{parskip} 	

%other packages for formatting
\RequirePackage{color}
\RequirePackage{graphicx}
\usepackage[usenames,dvipsnames]{xcolor}
\usepackage[scale=0.9]{geometry}

%tabularx environment
\usepackage{tabularx}

%for lists within experience section
\usepackage{enumitem}

% centered version of 'X' col. type
\newcolumntype{C}{>{\centering\arraybackslash}X} 

%to prevent spillover of tabular into next pages
\usepackage{supertabular}
\usepackage{tabularx}
\newlength{\fullcollw}
\setlength{\fullcollw}{0.47\textwidth}

%custom \section
\usepackage{titlesec}				
\usepackage{multicol}
\usepackage{multirow}

%CV Sections inspired by: 
%http://stefano.italians.nl/archives/26
\titleformat{\section}{\Large\scshape\raggedright}{}{0em}{}[\titlerule]
\titlespacing{\section}{0pt}{10pt}{10pt}

%for publications
\usepackage[style=authoryear,sorting=ynt, maxbibnames=2]{biblatex}

%Setup hyperref package, and colours for links
\usepackage[unicode, draft=false]{hyperref}
\definecolor{linkcolour}{rgb}{0,0.2,0.6}
\hypersetup{colorlinks,breaklinks,urlcolor=linkcolour,linkcolor=linkcolour}
\addbibresource{citations.bib}
\setlength\bibitemsep{1em}

%for social icons
\usepackage{fontawesome5}

%debug page outer frames
%\usepackage{showframe}

%----------------------------------------------------------------------------------------
%	BEGIN DOCUMENT
%----------------------------------------------------------------------------------------
\begin{document}

% non-numbered pages
\pagestyle{empty} 

%----------------------------------------------------------------------------------------
%	TITLE
%----------------------------------------------------------------------------------------

% \begin{tabularx}{\linewidth}{ @{}X X@{} }
% \huge{Your Name}\vspace{2pt} & \hfill \emoji{incoming-envelope} email@email.com \\
% \raisebox{-0.05\height}\faGithub\ username \ | \
% \raisebox{-0.00\height}\faLinkedin\ username \ | \ \raisebox{-0.05\height}\faGlobe \ mysite.com  & \hfill \emoji{calling} number
% \end{tabularx}

\begin{tabularx}{\linewidth}{@{} C @{}}
\Huge{Minsung Cho} \\[7.5pt]
\href{mailto:minsung@ccs.neu.edu}{\raisebox{-0.05\height}\ minsung@ccs.neu.edu} \ $|$ \ \href{cho.minsung.pl}{cho.minsung.pl} 
\end{tabularx}

%----------------------------------------------------------------------------------------
%	EDUCATION
%----------------------------------------------------------------------------------------
\section{Education}
\begin{tabularx}{\linewidth}{@{}l X@{}}	
\textbf{Northeastern University}, Ph.D. Computer Science & \hfill 2022--Present \\
[3pt]
Advisor: Steven Holtzen\vspace{0.5em}\\
\textbf{Carnegie Mellon University}, B.S. Mathematics and Philosophy (Logic track) & \hfill 2018--2022 \\ 
[3pt]
Thesis: \textit{Cops and Robbers in Lean} \\

Advisor: Jeremy Avigad

\end{tabularx}

%Experience
\section{Experience}

\begin{tabularx}{\linewidth}{ @{}l r@{} }
\textbf{NSF REU Researcher}, University of Tennessee at Chattanooga & \hfill 2021 \\[3pt]
\multicolumn{2}{@{}X@{}}{
\begin{minipage}[t]{\linewidth}
    \begin{itemize}[nosep,after=\strut, leftmargin=1em, itemsep=3pt]
        \item[-] Classified the Krein--von Neumann extension on regular even--order quasi--differential operators
        \item[-] Published in \textit{Opuscula Mathematica}
    \end{itemize}
    \end{minipage}
}
\end{tabularx}

\begin{tabularx}{\linewidth}{ @{}l r@{} }
\textbf{Researcher in Combinatorics}, Carnegie Mellon University & \hfill 2020 \\[3pt]
\multicolumn{2}{@{}X@{}}{
\begin{minipage}[t]{\linewidth}
    \begin{itemize}[nosep,after=\strut, leftmargin=1em, itemsep=3pt]
        \item[-] Generalized the cop-win property to 1-connected infinite graphs
        \item[-] Research featured on 2021 CMU Mathematics newsletter
        \item[-] Grant proposal featured by CMU Undergraduate Research for exceptional writing
    \end{itemize}
    \end{minipage}
}
\end{tabularx}

%Pubs
\section{Publications}

\textit{The Krein-von Neumann extension of a regular even order quasi-differential operator.} 
M. Cho, S. Hoisington, B. Udall, R. Nichols. Opuscula Mathematica. 41.6 (2021): 805-841.\footnote{In math, author order is alphabetical. All authors contributed equally.}

\section{Invited Talks}

\begin{tabularx}{\linewidth}{ @{}l r@{} }
    \textbf{Scaling Decision--Theoretic Probabilistic Programming Through Factorization} & \hfill \begin{tabular}{r}
        DRAGSTERS \\ @ PLDI 2023
    \end{tabular}\\[3pt]
    Joint work with Steven Holtzen. Also presented at PLDI SRC 2023.
\end{tabularx}

%Projects
\section{Projects}

\begin{tabularx}{\linewidth}{ @{}l r@{} }
    \textbf{Generic Max-Sum Optimization Problem Solving On Semirings} & \hfill Rust\\[3pt]
    \multicolumn{2}{@{}X@{}}{We introduce a class of semirings that admits a tractable branch-and-bound algorithm in the style of Huang and Darwiche to solve max-sum optimization problems frequently seen in probabilistic reasoning.}  \\
\end{tabularx}

\begin{tabularx}{\linewidth}{ @{}l r@{} }
    \textbf{The Next 700 Probabilistic Programming Languages Beyond Inference} & \hfill \\[3pt]
    \multicolumn{2}{@{}X@{}}{We discuss, on top of a simple monadic discrete probabilistic programming language, how optimization over inference can also take on (co-)monadic shape while still maintaining tractability via Boolean compilation.}  \\
\end{tabularx}

\begin{tabularx}{\linewidth}{ @{}l r@{} }
    \textbf{The Software Evolution of Theorem Provers} & \hfill Jupyter Notebook, Typescript\\[3pt]
    \multicolumn{2}{@{}X@{}}{We investigate the history of different theorem provers through a software engineering lens to visualize the development of the modern theorem proving community, focusing on Isabelle, Coq, and Lean.}  \\
\end{tabularx}

\begin{tabularx}{\linewidth}{ @{}l r@{} }
\textbf{Cops and Robbers in Lean} & \hfill Lean 3\\[3pt]
\multicolumn{2}{@{}X@{}}{We formalized the game of Cops and Robbers on a graph and associated theorems such as \textit{a complete graph is always cop-win} and \textit{every cop-win graph has a corner}.}  \\
\end{tabularx}

\section{Awards}

PLMW@PLDI 2023 scholarship recipient.

University and College honors from Carnegie Mellon.

\section{Skills}

Fluent in Korean and English.

Experience in functional programming (Lean, Coq, Haskell, ML dialects), Python, Rust, Dafny, LaTeX.

5 semesters of TA experience in mathematics and logic, including one graduate course, at CMU.
\end{document}
